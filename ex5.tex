\chapter{Change Directory (cd)}

\section{Do This}

I'm going to give you the instructions for these sessions one more time:

\begin{enumerate} 
\item You do \emph{not} type in the \verb|$| (unix) or \verb|>| (Windows).
\item You type in the stuff after this, then hit enter.  If I have \verb|$ cd temp| you just type \verb|cd temp| and hit enter.
\item The output comes after you hit enter, followed by another \verb|$| or \verb|>| prompt.
\end{enumerate}

\begin{code}{Linux/Mac OSX Exercise 5}
<< d['code/ex5.sh-session|pyg|l'] >>
\end{code}

\begin{code}{Windows Exercise 5}
<< d['code/ex5-win.sh-session|pyg|l'] >>
\end{code}

\section{You Learned This}

You made all these directories in the last exercise, and now you're just moving around inside them with the
\program{cd} command.  In my session above I also use \program{pwd} to check
where I am, so remember not to type the output that \program{pwd} prints.
For example, on line 3 you see \verb|~/temp| but that's the output of \program{pwd}
from the prompt above it.  \emph{Do not type this in}.

You should also see how I use the \verb|..| to move "up" in the tree and path.



\section{Do More}

A very important part of learning to the use command line interface (CLI) on a computer with a
graphical user interface (GUI) is figuring out how they work together.  When I started using
computers there was no "GUI" and you did everything with the DOS prompt (the CLI).  Later, when
computers became powerful enough that everyone could have graphics, it was simple for me
to match CLI directories with GUI windows and folders.

Most people today, however, have no comprehension of the CLI, paths, and directories.
In fact, it's very difficult to teach it to them and the only way to learn about the
connection is for you to constantly work with the CLI until one day it clicks that 
things you do in the GUI will show up in the CLI.

The way you do this is by spending some time finding directories with your GUI file browser, 
and then going to them with your CLI.  This is what you'll do next.

\begin{enumerate}
\item cd to the \file{joe} directory with one command.
\item cd back to \file{temp} with one command, but not further above that.
\item Find out how to cd to your "home directory" with one command.
\item cd to your Documents directory, then find it with your GUI file browser (Finder, Windows Explorer, etc.).
\item cd to your Downloads directory, then find it with your file browser.
\item Find another directory with your file browser, then cd to it.
\end{enumerate}

