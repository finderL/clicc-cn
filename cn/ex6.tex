\chapter{列出目录下的内容 (ls)}

\section{任务}

开始之前,确认你已经 \program{cd} 到了 temp 的上一级目录。如果你不确定现在
在哪个目录里,那就是用 \program{pwd} 找出来。

\begin{code}{Linux/Mac OSX 练习 6}
<< d['code/ex6.sh-session|pyg|l'] >>
\end{code}

\begin{code}{Windows 练习 6}
<< d['code/ex6-win.sh-session|pyg|l'] >>
\end{code}

\section{知识点}

\program{ls} 命令列出了你当前所在的目录下的内容。你看到我使用了 \program{cd}
变更目录,然后列出里边的内容,这样我就知道接下来该到哪个目录下面去了。

ls 命令有很多的选项,不过你后面会通过学习 \program{help} 来自己找到这些东西。

\section{更多任务}

\begin{enumerate}
\item \emph{输入每一条命令!} 要学习这些命令,你必须输入这些命令。\emph{光阅读是不够的}。
    这一点我以后就不跟你啰嗦了。
\item 如果你用 Unix,那就在 \file{temp} 目录中试一下 \verb|ls -lR| 命令。
\item Windows 下一样的功能可以通过 \verb|dir -R| 完成。
\item 使用 \program{cd} 进到别的目录下,然后通过 \program{ls} 看看里边有什么内容。
\item 在笔记本里记下你的问题。我知道你会有些问题,因为对于这条命令我并没讲全。
\item 记住如果你在路径中迷失了,就使用 \program{ls} 和 \program{pwd} 来找出你的
    当前路径,然后通过 \program{cd} 到达你的目的路径即可。
\end{enumerate}

