\chapter{复制文件 (cp)}

\section{任务}

\begin{code}{Linux/Mac OSX 练习 10}
<< d['code/ex10.sh-session|pyg|l'] >>
\end{code}

\begin{code}{Windows 练习 10}
<< d['code/ex10-win.sh-session|pyg|l'] >>
\end{code}

\section{知识点}

现在你学会了复制(copy)文件。很简单,就是把一个文件拷贝成一个新文件而已。在
这个练习中我还创建了一个新目录,然后将文件拷贝到其中去。

我现在要告诉你一个关于程序员和系统管理员的秘密:他们可懒了。我是懒人,我的朋友
们也是懒人,这也是我们用计算机的原因。我们让计算机为我们做各种无聊的事情。你学
到现在,所做的事情就是重复输入各种无趣的命令,并通过这个过程学会这些命令,但
实际工作中不是这样子的。在实际工作中,如果你发现某个任务需要通过无趣的重复工作
来完成,那么很可能已经有程序员找出让这个任务变得更简单的方法了,只不过你不知道
而已。

另外一件要告诉你的,就是程序员其实没有你想象的那么聪明。如果你过度思考要输入的
命令的名称,结果很可能是过而不及。取而代之的,你应该去想这个命令的名字,然后直接
试试这个名称或者类似这个名称的缩写。如果还是不灵,那就问问别人,或者上网搜索。
不过碰到 ROBOCOPY 这么二的命令名字那就谁也没辙了。

\section{更多任务}

\begin{enumerate}
\item 练习使用 \program{cp -r} 命令去复制一些包含文件的目录。
\item 将一个文件拷贝到你的 home 目录或者桌面。
\item 从图形界面找到你拷贝过的文件,然后用文本编辑器打开它们。
\item 有没有发现我有时会在目录的结尾放一个斜杠(slash) \verb|/| ?这样做的目的是
    保证输入的名称确实是一个目录,所以如果这个目录不存在,那么我就会看到一个错误
    信息。
\end{enumerate}

