\chapter{将来的路}

你已经完成了这本快速入门书。到这里你的水平基本上达到了能使用 shell 的程度了。
其实要学的技巧和命令用法还有很多,这里我会再给你一些最后的阅读和研究方向。

\section{Unix Bash 参考资料}

你一直在用的 shell 叫做 Bash。它不见得是最牛的 shell,不过你在到处都能看到它,
而且它也有不少的功能,所以作为入门是很不错的。接下来我向你提供一些关于 Bash 的
链接,你应该好好阅读一下:

\begin{description}
\item[Bash Cheat Sheet] \href{http://cli.learncodethehardway.org/bash_cheat_sheet.pdf}{http://cli.learncodethehardway.org/bash\_cheat\_sheet.pdf},作者 \href{http://freeworld.posterous.com/65140847}{Raphael},CC
license。
\item[Reference Manual] \href{http://www.gnu.org/software/bash/manual/bashref.html}{http://www.gnu.org/software/bash/manual/bashref.html}
\end{description}


\section{PowerShell 参考资料}

Windows 下能用的也就只有 PowerShell 了。以下是关于 PowerShell 的一些有用链接:

\begin{description}
\item[Owner's Manual] \href{http://technet.microsoft.com/en-us/library/ee221100.aspx}{http://technet.microsoft.com/en-us/library/ee221100.aspx}
\item[Cheat Sheet] \href{http://www.microsoft.com/download/en/details.aspx?displaylang=en&id=7097}{http://www.microsoft.com/download/en/details.aspx?displaylang=en\&id=7097}
\item[Master Powershell] \href{http://powershell.com/cs/blogs/ebook/default.aspx}{http://powershell.com/cs/blogs/ebook/default.aspx}
\end{description}

\section{向前进}

从现在开始你应该不再害怕使用命令行了。如果你立志要做一名程序员,本书算是一个
最好的起点,你可以领会到计算机其实就是一个“语言机器”,而 shell 则类似于一个
很迷你很易学的编程语言。

要想成为命令行高手,我的建议是你要强迫自己每天都使用它,不管用起来有多难。比较
困难的一点是在没有视觉提示的前提下要记住一些命令。图形界面的好处在于你可以通过
图形提示得到一些线索,从而知道工具该怎么使用。而命令行界面下,你就不得不从零开始
挖掘命令的功能,一开始你会觉得这件事情有些恼人的。

不过我有一个技巧可以减少你的痛苦。为自己创建一个小抄,写下你经常用到的命令行
技巧。使用命令行去做事情,如果你碰到一个会多次用到的命令,那就把它写下来。下次
你要用这个命令时,看一下你的小抄,这样你就会慢慢记住这些命令了。

最终你就用不到这样的小抄了。其实对我来说每天用到的命令也不过只有十来个,大部分
已经包含在这本书里边了。要记住十来个命令不是什么难事,所以没有什么能阻挡你的。

如果你遇到困难,你可以通过 help@learncodethehardway.org 联系我,我会帮你解决。
