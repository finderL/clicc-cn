\chapter{查看文件内容 (less, MORE)}

要完成这个练习,你将用到你学过的一些命令,另外你还需要一个文本编辑器来创建
纯文本(.txt)文件,以下是你要做的准备工作:

\begin{enumerate}
\item 打开文本编辑器,在新文本中输入一些东西。在 OSX 中你可以用
    TextWrangler,在 Windows 下你可以使用 Notepad++,在 Linux 中你可以用 
    Gedit,随便什么编辑器都可以。
\item 保存该文件到桌面,将其命名为 \file{ex12.txt}。
\item 在命令行(shell)使用你学过的命令将该文件拷贝(copy)到你的工作目录,也就是
    \file{temp} 目录中去。
\end{enumerate}

做好准备工作以后,就可以完成任务了。

\section{任务}

\begin{code}{Linux/Mac OSX 练习 12}
<< d['code/ex12.sh-session|pyg|l'] >>
\end{code}

就是这样子。要推出 \program{less},只要输入 \verb|q| 即可。这个 \verb|q| 
的意思是推出(quit)。

\begin{code}{Windows 练习 12}
<< d['code/ex12-win.sh-session|pyg|l'] >>
\end{code}

\begin{aside}{"displays file here"表示略掉了输出内容}
上面的输出中我用 \verb|[displays file here]| 来指代程序的输出。在后面的练习中,
如果碰到复杂情况无法向你展示输出内容,我就会用这个来指代你的输出。你的屏幕不会
显示这句话。
\end{aside}

\section{知识点}

这是查看文件内容的一个方法。它有用的地方在于,如果文件内容有很多行,它会将其
分页,这样就会每次显示一页。在“更多任务”中你会看到更多相关的练习。

\section{更多任务}

\begin{enumerate}
\item 再次打开你的文本文件,重复复制粘贴若干次,让你的文本长度约等于 50 至 100 
    行。
\item 将它再次复制到 \file{temp} 目录下,这样你就可以通过命令行查看了。
\item 再做一遍练习,不过这次你要逐页浏览文档。在 Unix 下使用空格键和 \verb|w| 键
    上下翻页,使用方向键也可以,不过在 Windows 下你就只能用空格键向下逐页浏览了。
\item 查看你创建的空文件的内容。
\item \program{cp} 命令会覆盖已经存在的文件,复制文件时要留意这一点。
\end{enumerate}

