\chapter{寻找文件 (find, DIR -R)}

\section{任务}

这个练习将三个概念合并到一条命令中来。我将展示给你如何找到你所有的文本文件,并
主页浏览它们。

\begin{code}{Linux/Mac OSX 练习 17}
<< d['code/ex17.sh-session|pyg|l'] >>
\end{code}

\begin{code}{Windows 练习 17}
<< d['code/ex17-win.sh-session|pyg|l'] >>
\end{code}

\section{知识点}

你学会了用 \program{find} (Windows 的 \program{dir -r}) 命令来搜索所有以  \file{.txt} 
结尾的文件,然后用 \program{less} (Windows 的 \program{more}) 来查看结果。

以下是 Unix 下的详解:

\begin{enumerate}
\item 首先我进到 \file{temp} 目录下。
\item 然后我简单执行了寻找(find)命令,因为这里也没有多少 \file{.txt} 文件。这个
    命令跟一句话差不多:“嘿,\program{find},从这里(.)开始,找到所有叫做"*.txt"
    的文件,然后将它们打印出来。”将命令用跟计算机对话的方式记下来,这是一个不错的
    记忆方法。
\item 接下来我到上一层目录执行了相同的命令,不过这次我将输出转向到了 \program{less} 
    中。这条命令执行起来可能会花一些时间,需要你耐心等待一下。
\item 然后我又执行了一遍,不过这次的时间可能会真的有点长,所以你可以随时用  CTRL-c 
    把它中断掉。
\end{enumerate}

Windows 下也差不多:

\begin{enumerate}
\item 然后我简单执行了 DIR 命令,因为这里也没有多少 \file{.txt} 文件。这条命令的
    意思是“列出当前目录以及所有子目录中的所有东西”。这是一个循环命令(recursive command)。
\item 接下来我到上一层目录执行了同样的命令,不过这次我把输出转向到 \program{more}。
    这条命令时间会有点长,需要耐心等待。
\item 然后我又执行了一遍,不过这次的时间可能会真的有点长,所以你可以随时用  CTRL-c 
    把它中断掉。
\end{enumerate}

\section{更多任务}

\begin{enumerate}
\item Unix:找出你的 \program{find} 索引卡,在描述面写下“find STARTDIR -name WILDCARD -print”。
    记住这个命令的格式,用我给你的句式辅助记忆。
\item Windows:同上,不过要写的是“dir -r -filter”。
\item 你可以将 \verb|.| 替换为任何目录,试着换一个目录搜索一下。
\item 寻早你计算机中所有的视频文件,从 home 目录开始找起,将输出用 \verb|>| 保存
    到一个文件中。记住你可以用 \verb|SOMECOMMAND > SOMEFILE.txt| 这样的格式,它
    会将 SOMECOMMAND 的输出写入到 SOMEFILE.txt 中。
\end{enumerate}

