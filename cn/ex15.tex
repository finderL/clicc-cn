\chapter{管道和重定向}

\section{任务}

\begin{code}{Linux/Mac OSX 练习 15}
<< d['code/ex15.sh-session|pyg|l'] >>
\end{code}

\begin{code}{Windows 练习 15}
<< d['code/ex15-win.sh-session|pyg|l'] >>
\end{code}

\section{知识点}

现在我们将接触命令行比较酷的知识点:重定向(redirection)。这个概念的意义在于
你可以修改程序的输入和输出的走向,你要通过  \verb|<| (小于号)、\verb|>| (大于号)、
\verb,|, (管道符)来做到这些。以下是详细解说:

\begin{description}
\item[$|$] \verb,|, 将左边命令的输出导向到右边命令中去。第 1 行向你演示了这一点。
\item[$<$] \verb|<| 将右边的文件作为输入发送给左边的程序。你看到第 2 行所做的就
   是这个。\emph{PowerShell 不支持这种操作。}
\item[$>$] \verb|>| 将左边命令的输出写入到右边的文件中去。第 9 行展示了这一点。
\item[$>>$] \verb|>>| 将左边命令的输出\emph{追加(append)}到右边的文件中去。我在
   第 9 行做了这样的操作。
\end{description}

还有一些别的符号,不过现在我们只学这些就够用了。

\section{更多任务}

\begin{enumerate}
\item 制作一些索引卡用来辅助记忆这些符号。将符号写在一侧,它的意义写在另一侧,
    像记忆命令一样将这些符号记下来。
\end{enumerate}

