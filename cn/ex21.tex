\chapter{环境变量 (env, echo, Env:)}

\section{任务}

\begin{code}{Linux/Mac OSX 练习 21}
<< d['code/ex21.sh-session|pyg|l'] >>
\end{code}

\begin{code}{Windows 练习 21}
<< d['code/ex21-win.sh-session|pyg|l'] >>
\end{code}

\section{知识点}

你的命令行(shell)有一些“隐藏的变量”,它们可以改变程序执行的效果。这个练习中我
首先打印出了我的环境变量,将它们输出到屏幕,然后我又执行了一遍,将输出转向到 
\program{grep} 并找到了包含我的用户名的环境变量。最后,我将一个叫做 TESTING 的
环境变量的值设为 "1 2 3"。

这个你也许不会经常用到,不过有时你要配置某些东西时会要修改这些环境变量。PATH 
变量就是一个很好的例子,它决定了系统是按怎样的顺序搜索到你要执行的命令的。后面的
练习中你将学到如何修改 PATH。


\section{更多任务}

\begin{enumerate}
\item 我要求你上网搜索,研究一下如何修改你的计算机中的 PATH。试着只用命令行来完成
    这个任务。
\end{enumerate}

