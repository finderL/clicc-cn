\chapter{在多个目录中切换 (pushd, popd)}

\section{任务}

\begin{code}{Linux/Mac OSX 练习 8}
<< d['code/ex8.sh-session|pyg|l'] >>
\end{code}

\begin{code}{Windows 练习 8}
<< d['code/ex8-win.sh-session|pyg|l'] >>
\end{code}

\section{知识点}

如果你用到这些命令,那你就非常接近编程领域了,不过这些命令非常好用,所以我非教你
不可。这些命令可以让你临时跑到某个不同的目录中,然后再回到你之前的目录,并且方便
地在两者之间切换。

\program{pushd} 命令会将你目前所在的目录“推送(push)”到一个列表中以供后续使用,
然后让你\emph{转入}到另一个目录中。它的意思大致是:“记住我的现在位置,然后到这个地方去。”

\program{popd} 命令会将你上次 push 过的目录从列表中“弹出(pop)”,然后让你回到
这个被“弹出”的目录。

最后,在 Unix 中的 \program{pushd} 有点不同,如果你运行时不添加任何参数,那么
它就会让你再当前目录和你上一次 push 过的目录之间切换,这个方法可以让你很方便地
在两个目录之间切换。\emph{不过 PowerShell 中这样做是不灵的。}


\section{更多任务}

\begin{enumerate}
\item 使用这些命令在你计算机目录之间多切换几次。
\item 删掉 \file{i/like/icecream} 这些目录,然后自己创建一些目录,在它们之间切换。
\item 向自己解释 \program{pushd} 和 \program{popd} 的输出的意义。有没有发现它的
    工作模式有点像一个堆栈。
\item 前面已经教过了,记住 \verb|mkdir -p| 会创建一个完整的多层目录,即使中间
    目录不存在也能成功。这也是我创建本习题一开始所做的事情。
\end{enumerate}

