\chapter{通配符匹配}

\section{任务}

\begin{code}{Linux/Mac OSX 练习 16}
<< d['code/ex16.sh-session|pyg|l'] >>
\end{code}

\begin{code}{Windows 练习 16}
<< d['code/ex16-win.sh-session|pyg|l'] >>
\end{code}

\section{知识点}

有时你需要用一条命令处理一批文件。处理的方法是使用星号 \verb|*| 来表示“任何文件”。
每当你使用了星号,命令行将会对匹配到非星号字符的文件生成一个列表。

这个练习中你用了各种方法列出你生成的文件。我的目录里还有一些多余的文件,你也许
也有一些。关键点在于,当你写下 \verb|*.txt| 时,你的意思是说“任何以 .txt 结尾
的文件。”

最后我们用 \verb|rm *.txt| 来删除 \file{temp} 目录下的所有 .txt 文件。

\section{更多任务}

\begin{enumerate}
\item 将 \verb|*| 加到你的速记卡中。在背后写下“通配符,用来匹配任何内容,例如 *.txt”。
\end{enumerate}

