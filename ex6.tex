\chapter{List Directory (ls)}

\section{Do This}

Before you start, make sure you \program{cd} back to the directory above temp.
If you have no idea where you are, use \program{pwd} to figure it out and then
move there.

\begin{code}{Linux/Mac OSX Exercise 6}
<< d['code/ex6.sh-session|pyg|l'] >>
\end{code}

\begin{code}{Windows Exercise 6}
<< d['code/ex6-win.sh-session|pyg|l'] >>
\end{code}

\section{You Learned This}

The \program{ls} command lists out the contents of the directory you
are currently in.  You can see me use \program{cd} to change into different
directories and then list what's in them so I know which directory to go to next.

There are a lot of options for the ls command, but you'll learn how to get
help on those later when we cover the \program{help} command.

\section{Do More}

\begin{enumerate}
\item \emph{Type every one of these commands in!} You have to actually type these
    to learn them.  Just reading them is \emph{not} good enough.  I'll stop yelling
    now.
\item On Unix, try the \verb|ls -lR| command while you're in \file{temp}.
\item On Windows do the same thing with \verb|dir -R|.
\item Use \program{cd} to get to other directories on your computer then use \program{ls} to see what's in them.
\item Update your notebook with new questions.  I know you probably have some, because I'm
    not covering everything about this command.
\item Remember that if you get lost, then use \program{ls} and \program{pwd} to figure out where you are, then go to where you need to be with \program{cd}.
\end{enumerate}

